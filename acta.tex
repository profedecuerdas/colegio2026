\documentclass[11pt]{article}
\usepackage[spanish]{babel}
\usepackage[utf8]{inputenc}
\usepackage[T1]{fontenc}
\usepackage{geometry}
\geometry{a4paper, margin=2.5cm}
\usepackage{graphicx}
\usepackage{fancyhdr}
\usepackage{tabularx}
\usepackage{booktabs}
\usepackage{multirow}
\usepackage{longtable}
\usepackage{xcolor}
\usepackage{hyperref}
\usepackage{datetime}

\title{Acta de Trabajo - Gestión Directiva}
\author{Institución Educativa Remigio Antonio Cañarte}
\date{16 y 19 de enero de 2026}

\pagestyle{fancy}
\fancyhf{}
\fancyhead[L]{\includegraphics[width=2cm]{logo.png}} % Reemplazar con logo si existe
\fancyhead[C]{\textbf{Acta de Trabajo - Gestión Directiva}}
\fancyhead[R]{\today}
\fancyfoot[C]{\thepage}

\begin{document}

\maketitle

\section*{Datos Generales}

\begin{tabular}{ll}
\textbf{Fecha de la reunión:} & 16 y 19 de enero de 2026 \\
\textbf{Modalidad:} & 16/01: Virtual; 19/01: Presencial \\
\textbf{Lugar:} & Sede El Rocío (19 de enero) \\
\textbf{Hora inicio:} & 16/01: Por definir; 19/01: 8:45 a.m. \\
\textbf{Hora finalización:} & 16/01: 1:00 p.m.; 19/01: 10:00 a.m. \\
\textbf{Convocante:} & Juan Manuel Arbeláez Castro (Rector) \\
\textbf{Secretario:} & Jefferson Jhoan Soto G. \\
\end{tabular}

\section*{Participantes}
\begin{itemize}
\item Jefferson Jhoan Soto G.
\item Ruby Vélez Corrales
\item Adriana Patricia Valbuena Bulla
\item María Stella López Pérez
\item Mauricio Rojas Henao
\item Sandra Milena Piedrahita Toloza
\item Carmen Ligia Mosquera Jaramillo
\item Martha Belén Ocampo Hincapié
\item Zulma Constanza López
\item Luz Bibiana Giraldo
\item María Mercedes Ramírez Serna
\item Alba Nelly Ramírez Serna
\item Anyele Loaiza Loaiza
\item Luz Yaneth Rodríguez Fernández
\item Juan Manuel Arbeláez Castro (Director)
\end{itemize}

\section*{Objetivos de la Reunión}
\begin{enumerate}
\item Fortalecer la estrategia pedagógica institucional mediante la definición, apropiación y seguimiento de un modelo pedagógico coherente, inclusivo y orientado al desarrollo de competencias.
\item Consolidar prácticas pedagógicas innovadoras que fortalezcan la motivación, el aprendizaje significativo y la equidad.
\end{enumerate}

\section*{Desarrollo de la Jornada}

\subsection*{16 de enero de 2026 (Modalidad Virtual)}
\begin{enumerate}
\item Se llamó a lista y se verificó la asistencia de los participantes.
\item Se compartió el enlace del documento colaborativo:\\
\url{https://docs.google.com/spreadsheets/d/1LHeS5u0m1sZvIt3SRdHYVIjXRHdYz9zV/edit?hl=es&gid=1847131143}
\item Se concentró el trabajo en la construcción de la \textbf{Estrategia Pedagógica Institucional}.
\item Se propuso la metodología de trabajo:
\begin{itemize}
\item Leer la estrategia pedagógica existente
\item Revisar el PMI (Plan de Mejoramiento Institucional)
\item Analizar los 4 elementos de la estrategia pedagógica
\end{itemize}
\item Se investigaron y compartieron conceptos sobre estrategia pedagógica, llegando a consensos sobre:
\begin{itemize}
\item Fortalecer el proceso enseñanza-aprendizaje mediante la implementación de una estrategia pedagógica integral que promueva el desarrollo académico, social y ético.
\item Atender a la diversidad y demandas del contexto institucional y social.
\item Desarrollar competencias académicas, sociales y emocionales.
\end{itemize}
\item Se dedicó parte de la reunión al diligenciamiento de la columna J (Actividades) del documento PMI Directiva.
\item A la 1:00 p.m., el Rector dio instrucciones para que cada participante:
\begin{itemize}
\item Pensara en actividades específicas
\item Revisara documentación sobre estrategias pedagógicas
\item Preparara aportes para la reunión presencial del 19 de enero
\end{itemize}
\item Se compartieron enlaces de referencia:
\begin{itemize}
\item \url{https://es.scribd.com/document/545619733/Estrategias-pedagogicas-y-estrategias-didacticas}
\item \url{https://psicologiaymente.com/desarrollo/estrategias-ensenanza}
\end{itemize}
\end{enumerate}

\subsection*{19 de enero de 2026 (Modalidad Presencial)}
\begin{enumerate}
\item Se retomó la reunión a las 8:45 a.m. en la Sede El Rocío.
\item Se trabajó de manera grupal bajo la dirección del Rector en la \textbf{PMI Directiva}.
\item Se avanzó en las siguientes columnas:
\begin{itemize}
\item Columna I: Actividades
\item Columna F: Estrategia
\item Columna G: Meta
\end{itemize}
\item Se consolidaron las actividades propuestas en la reunión anterior.
\item Se terminó la jornada de trabajo a las 10:00 a.m.
\end{enumerate}

\section*{Acuerdos y Conclusiones}
\begin{enumerate}
\item Se consensuó la siguiente definición de estrategia pedagógica institucional:\\
\textit{"Fortalecer el proceso de enseñanza y aprendizaje mediante la consolidación de una estrategia pedagógica institucional coherente, inclusiva y participativa, orientada al desarrollo de competencias académicas, sociales, emocionales y éticas de los estudiantes, en atención a la diversidad y a las demandas del contexto institucional y social."}

\item Se estableció como enfoque pedagógico:\\
\textit{"Enfoque pedagógico por competencias, inclusivo y participativo, con liderazgo pedagógico y toma de decisiones basada en la reflexión sobre la práctica y el contexto."}

\item Se definieron las siguientes actividades prioritarias:
\begin{enumerate}
\item Realizar diagnóstico institucional sobre prácticas pedagógicas actuales y necesidades del contexto estudiantil.
\item Analizar referentes pedagógicos nacionales e institucionales acordes con la diversidad y el enfoque por competencias.
\item Construir de manera participativa la propuesta de modelo pedagógico institucional.
\item Socializar la propuesta del modelo pedagógico con los diferentes estamentos de la comunidad educativa.
\item Ajustar y formalizar el modelo pedagógico mediante acto administrativo institucional.
\item Incorporar el modelo pedagógico al PEI y documentos de gestión académica.
\item Diseñar e implementar un plan de acompañamiento pedagógico liderado por los directivos docentes.
\item Realizar jornadas pedagógicas para definir, unificar y consensuar la estrategia pedagógica institucional.
\end{enumerate}

\item Se acordó continuar el diligenciamiento del documento PMI Directiva en próximas reuniones.
\end{enumerate}

\section*{Documentos de Referencia}
\begin{enumerate}
\item Documento colaborativo PMI Directiva:\\
\url{https://docs.google.com/spreadsheets/d/1-vyVPrmPwbiw6T34KGbOe1KDDUQ1xLn1/edit?gid=1993062863}
\item Documento de autoevaluación:\\
\url{https://docs.google.com/spreadsheets/d/1-vyVPrmPwbiw6T34KGbOe1KDDUQ1xLn1/edit?usp=sharing}
\end{enumerate}

\section*{Compromisos}
\begin{tabularx}{\textwidth}{|l|X|l|}
\hline
\textbf{Responsable} & \textbf{Actividad} & \textbf{Fecha} \\
\hline
Equipo Directivo & Consolidar documento PMI Directiva con actividades definidas & 26 de enero 2026 \\
\hline
Rector & Socializar avances con comunidad educativa & Febrero 2026 \\
\hline
Todos los participantes & Revisar y aportar a la estrategia pedagógica & Permanente \\
\hline
\end{tabularx}

\section*{Anexos}
\begin{enumerate}
\item Avances del documento PMI Directiva al 19 de enero de 2026 (9:16 a.m.)
\item Propuestas de estrategia pedagógica compartidas en el chat
\end{enumerate}

\vspace{2cm}

\begin{flushright}
\begin{tabular}{c}
\hline
\textbf{Juan Manuel Arbeláez Castro} \\
Rector - Institución Educativa Remigio Antonio Cañarte \\
\end{tabular}
\end{flushright}

\begin{flushright}
\begin{tabular}{c}
\hline
\textbf{Jefferson Jhoan Soto G.} \\
Secretario - Acta de Trabajo \\
\end{tabular}
\end{flushright}

\section*{Distribución}
\begin{itemize}
\item Archivo Institucional
\item Participantes de la reunión
\item Consejo Directivo
\item Coordinadores Académicos
\end{itemize}

\end{document}
