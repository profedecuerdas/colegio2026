\documentclass[11pt, letterpaper]{article}

% --- Paquetes de Idioma y Codificación ---
\usepackage[utf8]{inputenc}
\usepackage[T1]{fontenc}
\usepackage[spanish, es-tabla]{babel}

% --- Diseño de Página ---
\usepackage[margin=2.5cm]{geometry}
\usepackage{titlesec}
\usepackage{enumitem}
\usepackage{parskip}

% Configuración de títulos
\titleformat{\section}{\normalfont\large\bfseries}{}{0pt}{\underline}

\begin{document}

% --- Encabezado ---
\begin{center}
    {\Large \textbf{I.E. REMIGIO ANTONIO CAÑARTE}} \\
    {\large \textbf{ACTA DE ACTUALIZACIÓN DEL PMI 2026}} \\
    \vspace{0.4cm}
    \textbf{GESTIÓN:} Directiva
\end{center}

\vspace{0.5cm}

% --- Información de las Sesiones ---
\section*{1. DATOS DE LAS JORNADAS}
\begin{description}
    \item[Sesión 1 (Virtual):] Viernes, 16 de enero de 2026 (Google Meet).
    \item[Sesión 2 (Presencial):] Lunes, 19 de enero de 2026 (Sede Principal).
    \item[Presidente de la sesión:] Juan Manuel Arbeláez Castro (Rector).
    \item[Responsable del Acta:] Jefferson Jhoan Soto G.
\end{description}

\section*{2. INTEGRANTES DEL EQUIPO DE TRABAJO}
\begin{itemize}[label=--, itemsep=0pt]
    \item Jefferson Jhoan Soto G.
    \item Ruby Vélez corrales
    \item Adriana Patricia Valbuena Bulla
    \item María Stella López Pérez
    \item Mauricio Rojas Henao
    \item Sandra Milena Piedrahita Tolosa
    \item Carmen Ligia Mosquera Jaramillo
    \item Martha Belén Ocampo Hincapié
    \item Zulma Constanza López
    \item Luz Bibiana Giraldo
    \item María Mercedes Ramírez Serna
    \item Alba Nelly Ramírez serna
    \item Anyele Loaiza Loaiza
    \item Luz Yaneth Rodríguez Fernández
    \item Juan Manuel Arbeláez Castro
\end{itemize}

\section*{3. DINÁMICA DE TRABAJO Y APORTES}
Según lo registrado en la jornada, el proceso se desarrolló así:
\begin{itemize}[label=$\bullet$]
    \item \textbf{16 de enero (Virtual):} Se debatió la definición de Estrategia Pedagógica buscando una "expresión propia". Se acordó que la Misión y Visión deben responder a 4 ejes: Pedagógico, Tecnológico, Ambiental y Derechos Humanos. Se inició la redacción grupal de la Columna I (Actividades).
    \item \textbf{19 de enero (Presencial):} Se trabajó en la construcción de las Columnas F (Estrategia) y G (Meta), consolidando la información hasta el cierre de la jornada.
\end{itemize}

\section*{4. CONTENIDO TÉCNICO CONSOLIDADO (HOJA PMI DIRECTIVA)}
Se transcribe el contenido literal de la matriz para el componente de Estrategia Pedagógica:

\textbf{PROBLEMA MEJORAR/ PRIORIDAD (Columna D):} \\
1. Inexistencia del modelo pedagogico debe estar claramente definido coherente con la mision y vision. (Conocimiento del contexto y del estudiante). y así Garantizar la atencion a la diversidad y a la inclusion (Diferentes estilos y ritmos de aprendizaje). \\
2. Falta de unificación de los docentes para conocer, comprender y compartir la estrategia pedagogica, para una mayor apropiacion de la misma.

\textbf{OBJETIVO ( ¿Que queremos lograr? ) (Columna E):} \\
Fortalecer el proceso de enseñanza y aprendizaje mediante la consolidación de una estrategia pedagógica institucional coherente, inclusiva y participativa, orientada al desarrollo de competencias académicas, sociales, emocionales y éticas de los estudiantes, en atención a la diversidad y a las demandas del contexto institucional y social.

\textbf{ESTRATEGIA, META Y RECURSOS (Columnas F, G, J):} \\
A la fecha de cierre de esta acta, estos campos se encuentran en proceso de elaboración y revisión por parte del equipo directivo en el documento compartido.

\section*{5. CIERRE}
Siendo las **10:00 AM del lunes 19 de enero de 2026**, se da por terminada la jornada de trabajo con los avances registrados en la matriz institucional.

\vspace{2.5cm}

% --- Firmas ---
\begin{minipage}[t]{0.45\textwidth}
    \rule{\textwidth}{0.4pt} \\
    \textbf{Juan Manuel Arbeláez Castro} \\
    Rector
\end{minipage}
\hfill
\begin{minipage}[t]{0.45\textwidth}
    \rule{\textwidth}{0.4pt} \\
    \textbf{Jefferson Jhoan Soto G.} \\
    Responsable del Acta
\end{minipage}

\end{document}
