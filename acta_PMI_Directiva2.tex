\documentclass[11pt, letterpaper]{article}
\usepackage[utf8]{inputenc}
\usepackage[T1]{fontenc}
\usepackage[spanish, es-tabla]{babel}
\usepackage[margin=2.5cm]{geometry}
\usepackage{titlesec}
\usepackage{enumitem}
\usepackage{parskip}

% Esto evita que las comillas dobles generen errores como el de "expresión"
\shorthandoff{"}

\titleformat{\section}{\normalfont\large\bfseries}{}{0pt}{\underline}

\begin{document}

\begin{center}
    {\Large \textbf{I.E. REMIGIO ANTONIO CAÑARTE}} \\
    {\large \textbf{ACTA DE ACTUALIZACIÓN DEL PMI 2026}} \\
    \vspace{0.4cm}
    \textbf{GESTIÓN:} Directiva
\end{center}

\section*{1. DATOS DE LAS JORNADAS}
\begin{description}
    \item[Sesión 1 (Virtual):] Viernes, 16 de enero de 2026.
    \item[Sesión 2 (Presencial):] Lunes, 19 de enero de 2026. Cierre 10:00 AM.
    \item[Presidente:] Juan Manuel Arbeláez Castro (Rector).
    \item[Responsable del Acta:] Jefferson Jhoan Soto G.
\end{description}

\section*{2. INTEGRANTES DEL EQUIPO DE TRABAJO}
\begin{itemize}[label=--, itemsep=0pt]
    \item Jefferson Jhoan Soto G. \item Ruby Vélez corrales \item Adriana Patricia Valbuena Bulla
    \item María Stella López Pérez \item Mauricio Rojas Henao \item Sandra Milena Piedrahita Tolosa
    \item Carmen Ligia Mosquera Jaramillo \item Martha Belén Ocampo Hincapié \item Zulma Constanza López
    \item Luz Bibiana Giraldo \item María Mercedes Ramírez Serna \item Alba Nelly Ramírez serna
    \item Anyele Loaiza Loaiza \item Luz Yaneth Rodríguez Fernández \item Juan Manuel Arbeláez Castro
\end{itemize}

\section*{3. DINÁMICA DE TRABAJO}
\begin{itemize}[label=$\bullet$]
    \item \textbf{16 de enero:} Debate sobre Estrategia Pedagógica para establecer una ``expresión propia''. Se definieron los 4 ejes del horizonte (Pedagógico, Tecnológico, Ambiental y DDHH) y se redactó la Columna I.
    \item \textbf{19 de enero:} Elaboración de las Columnas F y G en la Sede Principal.
\end{itemize}

\section*{4. CONTENIDO TÉCNICO (HOJA PMI DIRECTIVA)}
\textbf{PROBLEMA MEJORAR/ PRIORIDAD (Columna D):} \\
1. Inexistencia del modelo pedagogico debe estar claramente definido coherente con la mision y vision (conocimiento del contexto y del estudiante), y así garantizar la atencion a la diversidad y a la inclusion. \\
2. Falta de unificación de los docentes para conocer, comprender y compartir la estrategia pedagogica.

\textbf{OBJETIVO (Columna E):} \\
Fortalecer el proceso de enseñanza y aprendizaje mediante la consolidación de una estrategia pedagógica institucional coherente, inclusiva y participativa.

\textbf{ESTRATEGIA, META Y RECURSOS (F, G, J):} \\
Campos en proceso de elaboración en el documento compartido al momento del cierre.

\section*{5. CIERRE}
Finalización: Lunes 19 de enero de 2026, 10:00 AM.

\vspace{2cm}
\noindent
\begin{minipage}[t]{0.45\textwidth}
    \rule{\textwidth}{0.4pt} \\ \textbf{Juan Manuel Arbeláez Castro} \\ Rector
\end{minipage}
\hfill
\begin{minipage}[t]{0.45\textwidth}
    \rule{\textwidth}{0.4pt} \\ \textbf{Jefferson Jhoan Soto G.} \\ Responsable del Acta
\end{minipage}

\end{document}
